%!TEX root=../paper.tex
\paragraph{ChaCha for secure communication}
Secure communication on the internet typically requires different cryptographic primitives and a common protocol applying these primitives to provide a protected channel between endpoints.
The Transport Layer Security (TLS) specifies the leading and standard protocols for secure communication.
The TLS protocol defines public key algorithms for establishing symmetric session keys, and different symmetric and MAC algorithms for the subsequent encrypted and authenticated communication.
The efficiency of these primitives is essential to achieve good performance for secure communication.
ChaCha is a high-throughput stream cipher which is a refinement of the Salsa20 stream cipher. It targets software platforms to aim at improving its security bounds without losing performance. ChaCha stream cipher and Poly1305 authenticator~\cite{RFC:16} are specified as one of the cipher suites by the current TLS 1.3~\cite[Section 9.1]{RFC:18:8446}.
ChaCha is officially supported by popular cryptographic/TLS projects like OpenSSL, OpenSSH, and MbedTLS.
Moreover, there are extensive efforts in the literature to intensively optimise the performance of ChaCha implementation on various platforms, namely optimised software (using ARM Cortex ISA~\cite{SSS:17} or RISC-V ISA~\cite{Sto:19}), vectorisation AVX architecture~\cite{GolGue:14}, and dedicated hardware accelerators~\cite{KLA:19,PRH:19}.  

\paragraph{RISC-V Instruction Set Architecture}
RISC-V is an open and free ISA specification with academic origins~\cite{riscv:14} adopting strongly RISC-oriented design principles. 
The ISA can be implemented, modified, or extended with neither licence nor royalty requirements.
As a result of these features and availability of supporting ecosystems (e.g., compilation tool-chains) from the surrounding community,
an increasing number of (typically open-source) RISC-V implementations have been available.
The RISC-V ISA is designed with 32 registers, denoted GPR[$i$] for $0 \le i < 32$: GPR[$0$] is fixed to 0, whereas GPR[$1$] to GPR[$31$] are general-purpose. 
The width of each GPR[$i$], and hence the base ISA are defined by \emph{XLEN} of which supported values can be 32, 64, 128 bits. 
RISC-V has an extremely simple ISA, consisting of about 50 general-purpose instructions, that has been designed to be extended.
Thanks to the simple base ISA, the implementation of a RISC-V processor even with 64-bit ISA can achieve low area cost 
that is well suitable for resource-constrained devices. 
For example, the S2 Series, developed by SiFive\footnote{https://www.sifive.com/cores/s21}, is a family of full-featured 64-bit RISC-V embedded processors for area-constrained applications.
The base ISA can be supplemented using sets of standard or non-standard extensions to support additional special-purposes.
Multiple proposals for standard extensions particularly one of which is the cryptography extension\footnote{https://lists.riscv.org/g/tech-crypto-ext} are currently being developed. The current cryptography extension proposal consists of three main components: Vector, Scalar, and Entropy Source instructions.
The Scalar component aiming at resource-constrained devices defines a set of algorithm-specific, e.g., AES, SM4, SHA2, instructions. 
However, ChaCha, a widely-used algorithm, is not explicitly supported by the current proposal. 
In this context, the paper proposes the first design and implementation of dedicated ISEs to accelerate the ChaCha algorithm.

\paragraph{Accelerate cryptographic implementation via an ISE approach}
There are various styles that often exist for implementing a given cryptographic algorithm. 
Techniques can be algorithm-agnostic or algorithm-specific, and based on the use of hardware only, software only, or a hybrid approach. 
ISE~\cite{GalBer:11,BarGioMar:09,RegIen:16}, as a hybrid approach, has proved its effectiveness within the context of cryptography. 
An increasing number of studies recently adopt ISEs for cryptographic application to improve efficiency~\cite{RCB:20,MNPSW:21} as well as address security concerns~\cite{GMPP:20}. 
For accelerating cryptographic implementation performance,
the idea is that a set of additional instructions can be, e.g., through benchmarking, identified 
to allow the cryptographic operation to leverage special-purpose functionality, 
vs. general-purpose functionality in the base ISA, and thereby deliver improvement wrt.\ pertinent quality metrics.
ISEs are {\em particularly} effective for resource-constrained (embedded) devices 
because they afford a compromise improving footprint and latency vs. a software-only option 
while also improving area overhead and flexibility vs. a hardware-only option.

\paragraph{Remit and organisation}
This paper investigates an ISE approach to support for ChaCha software. We favour a lightweight method to accelerate ChaCha performance targeting resource-constrained devices. The paper is organised as follows:
\REFSEC{sec:bg} provides some background and an abstract implementation of ChaCha.
\REFSEC{sec:ise} proposes the design and implementation of ISE variants for ChaCha. 
In \REFSEC{sec:res}, we first present the evaluation of ISE variants on ChaCha block function, then realise a complete ISE-assisted ChaCha implementation which is evaluated in comparison to optimised implementations using the RISC-V base ISA and vector extensions. 
Finally, \REFSEC{sec:outro} gives some conclusions.
