%!TEX root=../paper.tex
\subsection{Software Evaluation}

\paragraph{Single block function performance}
We firstly evaluate the performance of Chacha20's software block function using each proposed ISE variants in compared with a vanilla implementation (i.e., Baseline) based on the typical RV64IMA Architecture. The evaluation is shown in terms of Instruction count, Cycle count, and Instruction footprint of the Chacha20's block function.

\begin{table}
\caption{Comparison of chacha single block function performance}
\label{tab:res:sw:perf}
\begin{tabular}{lccc}
\toprule            
Implementations        & Instruction count   & Cycle count & Instruction footprint\\

\midrule
Baseline     & 2214     & 2803  &  852 \\
 + ISE $V_1$ &  434     & 1221  &  382 \\
 + ISE $V_2$ &  514     & 1597  &  428 \\
 + ISE $V_3$ &  594     & 1972  &  454 \\
 + ISE $V_4$ &  464     & 1818  &  274 \\

\bottomrule
\end{tabular}
\end{table}

\paragraph{Encryption/Decryption performance}
We then implement a completed Chacha20 encryption/decryption operations using the proposed ISE $V_4$ to compare with the current Chacha encryption in BoringSSL, as a Baseline, and RISC-V vector instruction extension assisted implementations.

\begin{table}
\caption{Comparison of encryption/decryption performance in instruction count for different message sizes between the Baseline, the proposed ISE and different vectorization implementations}
\label{tab:res:sw:perf}
\begin{tabular}{rcccccc}
\toprule             
Message size & Baseline  & \multicolumn{2}{c}{128 bits} & 256 bits  &   ISE       \\
             & BoringSSL & Vec $V_1$   &   Vec $V_2$    & Vec $V_1$ &   ISE $V_4$ \\
\midrule
  64 Bytes   &    2991   &    1998     &       654      &    1998   &   523       \\
 128 Bytes   &    5788   &    1998     &      1285      &    1998   &   989       \\
 256 Bytes   &   11382   &    1998     &      2547      &    1998   &  1921       \\
 512 Bytes   &   22768   &    3745     &      5071      &    1998   &  3785       \\
1024 Bytes   &   45347   &    7239     &     10119      &    3745   &  7716       \\
\bottomrule
\end{tabular}
\end{table}

\subsection{Hardware Evaluation}

\begin{table}
\caption{Comparison of hardware overheads between ISE variants when synthesised for a
generic CMOS cell library}
\label{tab:res:sw:hardcost1}
\begin{tabular}{lcc}
\toprule            
Implementations        & Size (NAND2 cells)    & Depth  \\

\midrule
           + ISE $V_1$ &     2353     & 56   \\
           + ISE $V_2$ &     2494     & 41   \\
           + ISE $V_3$ &     1362     & 25   \\
           + ISE $V_4$ &     1617     & 19   \\

\bottomrule
\end{tabular}
\end{table}

\begin{table}
\caption{Comparison of hardware overheads between sub-modules}
\label{tab:res:sw:hardcost2}
\begin{tabular}{lcc}
\toprule            
Algorithm        &     Size (NAND2 cells)     & Depth \\

\midrule
RV64 Rocket Core &    98150     &   91  \\
~~~~~|-- ALU     &     3719     &   28  \\
~~~~~|-- Muldiv  &    17171     &   40  \\
+ ISE $V_4$      &     1617     &   19  \\ 

\bottomrule
\end{tabular}
\end{table}

