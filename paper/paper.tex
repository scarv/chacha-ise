
%\documentclass[preprint]{iacrtrans}
\documentclass[sigconf]{acmart}

\usepackage{paper}


\begin{document}

% =============================================================================

\title{A lightweight ISE for ChaCha20 on 64-bit RISC-V}
%\keywords{}
\ifbool{anonymous}{%
\author{}
%\institute{}
}{%
\ifbool{ACMsubmission}{%
\author{Ben Marshall}
\affiliation{%
  \institution{Department of Computer Science, University of Bristol.}
  \country{}}%
\email{ben.marshall@bristol.ac.uk}
\author{Daniel Page}
\affiliation{%
  \institution{Department of Computer Science, University of Bristol.}
  \country{}}%
\email{daniel.page@bristol.ac.uk}
\author{Thinh Pham}
\affiliation{%
  \institution{Department of Computer Science, University of Bristol.}
\country{}}%
\email{th.pham@bristol.ac.uk}
}{%
\author{%
  Ben Marshall \and Dan Page \and Thinh Pham }%
\institute{%
  Department of Computer Science, University of Bristol.                  \\
  \email{{ben.marshall,daniel.page,th.pham}@bristol.ac.uk}}%
}%
}%

\ifbool{ACMsubmission}{%
}{%
\maketitle
}%

\begin{abstract}
The ChaCha20 is a high-throughput stream cipher designed with the aim of ensuring high-security margins while achieving high performance on software platforms. RISC-V, an emerging, free, and open Instruction Set Architecture (ISA) is being developed with many instruction set extensions (ISE). ISEs are a native concept in RISC-V to support a relatively small RISC-V ISA to suit different use-cases including cryptographic acceleration via either standard or custom ISEs. This paper proposes a lightweight Instruction Set Extension (ISE) to support the ChaCha20 on 64-bit RISC-V architectures targeting embedded, IoT class devices. The proposed ISE is designed to accelerate the computation of ChaCha20's block function and align with the  RISC-V design principles. We show that our proposed ISEs help to improve the efficiency of ChaCha block function significantly. 
The ISE-assisted implementation of ChaCha encryption speeds up at least $5.4\times$ and $3.4\times$ compared to the OpenSSL baseline and ISA-base optimised implementation, respectively. 
For encrypting short messages, the ISE-assisted implementation gains a comparative performance compared to the implementations using very high area overhead vector extensions.
\end{abstract}

\ifbool{ACMsubmission}{%
\maketitle
}%

% =============================================================================

%\tableofcontents

\section{Introduction}
\label{sec:intro}
%!TEX root=../paper.tex
\paragraph{ChaCha for secure communication}
Secure communication on the internet typically requires different cryptographic primitives and a common protocol applying these primitives to provide a protected channel between endpoints.
The Transport Layer Security (TLS) specifies the leading and standard protocols for secure communication.
The TLS protocol defines public key algorithms for establishing symmetric session keys, and different symmetric and MAC algorithms for the subsequent encrypted and authenticated communication.
The efficiency of these primitives is essential to achieve good performance for secure communication.
ChaCha is a high-throughput stream cipher which is a refinement of the Salsa20 stream cipher. It targets software platforms to aim at improving its security bounds without losing performance. ChaCha stream cipher and Poly1305 authenticator~\cite{RFC:16} are specified as one of the cipher suites by the current TLS 1.3~\cite[Section 9.1]{RFC:18:8446}.
ChaCha is officially supported by popular cryptographic/TLS projects like OpenSSL, OpenSSH, and MbedTLS.
Moreover, there are extensive efforts in the literature to intensively optimise the performance of ChaCha implementation on various platforms, namely optimised software (using ARM Cortex ISA~\cite{SSS:17} or RISC-V ISA~\cite{Sto:19}), vectorisation AVX architecture~\cite{GolGue:14}, and dedicated hardware accelerators~\cite{KLA:19,PRH:19}.  

\paragraph{RISC-V Instruction Set Architecture}
RISC-V is an open and free ISA specification with academic origins~\cite{riscv:14} adopting strongly RISC-oriented design principles. 
The ISA can be implemented, modified, or extended with neither licence nor royalty requirements.
As a result of these features and availability of supporting ecosystems (e.g., compilation tool-chains) from the surrounding community,
an increasing number of (typically open-source) RISC-V implementations have been available.
The RISC-V ISA is designed with 32 registers, denoted GPR[$i$] for $0 \le i < 32$: GPR[$0$] is fixed to 0, whereas GPR[$1$] to GPR[$31$] are general-purpose. 
The width of each GPR[$i$], and hence the base ISA are defined by \emph{XLEN} of which supported values can be 32, 64, 128 bits. 
RISC-V has an extremely simple ISA, consisting of about 50 general-purpose instructions, that has been designed to be extended.
Thanks to the simple base ISA, the implementation of a RISC-V processor even with 64-bit ISA can achieve low area cost 
that is well suitable for resource-constrained devices. 
For example, the S2 Series, developed by SiFive\footnote{https://www.sifive.com/cores/s21}, is a family of full-featured 64-bit RISC-V embedded processors for area-constrained applications.
The base ISA can be supplemented using sets of standard or non-standard extensions to support additional special-purposes.
Multiple proposals for standard extensions particularly one of which is the cryptography extension\footnote{https://lists.riscv.org/g/tech-crypto-ext} are currently being developed. The current cryptography extension proposal consists of three main components: Vector, Scalar, and Entropy Source instructions.
The Scalar component aiming at resource-constrained devices defines a set of algorithm-specific, e.g., AES, SM4, SHA2, instructions. 
However, ChaCha, a widely-used algorithm, is not explicitly supported by the current proposal. 
In this context, the paper proposes the first design and implementation of dedicated ISEs to accelerate the ChaCha algorithm.

\paragraph{Accelerate cryptographic implementation via an ISE approach}
There are various styles that often exist for implementing a given cryptographic algorithm. 
Techniques can be algorithm-agnostic or algorithm-specific, and based on the use of hardware only, software only, or a hybrid approach. 
ISE~\cite{GalBer:11,BarGioMar:09,RegIen:16}, as a hybrid approach, has proved its effectiveness within the context of cryptography. 
An increasing number of studies recently adopt ISEs for cryptographic application to improve efficiency~\cite{RCB:20,MNPSW:21} as well as address security concerns~\cite{GMPP:20}. 
For accelerating cryptographic implementation performance,
the idea is that a set of additional instructions can be, e.g., through benchmarking, identified 
to allow the cryptographic operation to leverage special-purpose functionality, 
vs. general-purpose functionality in the base ISA, and thereby deliver improvement wrt.\ pertinent quality metrics.
ISEs are {\em particularly} effective for resource-constrained (embedded) devices 
because they afford a compromise improving footprint and latency vs. a software-only option 
while also improving area overhead and flexibility vs. a hardware-only option.

\paragraph{Remit and organisation}
This paper investigates an ISE approach to support for ChaCha software. We favour a lightweight method to accelerate ChaCha performance targeting resource-constrained devices. The paper is organised as follows:
\REFSEC{sec:bg} provides some background and an abstract implementation of ChaCha.
\REFSEC{sec:ise} proposes the design and implementation of ISE variants for ChaCha. 
In \REFSEC{sec:res}, we first present the evaluation of ISE variants on ChaCha block function, then realise a complete ISE-assisted ChaCha implementation which is evaluated in comparison to optimised implementations using the RISC-V base ISA and vector extensions. 
Finally, \REFSEC{sec:outro} gives some conclusions.

% =============================================================================

\section{ChaCha20 stream cipher}
\label{sec:bg}
%!TEX root=../paper.tex
Brief introduction to Chacha20
\subsection{ChaCha's block function}
Describe what is block function and how it is computed

\subsection{ChaCha's round function}
Describe what is round function and how it is computed

\subsection{ChaCha's quarter round function}
Describe what is quarter round function and how it is computed

% =============================================================================

\section{Proposed ISEs for Chacha20}
\label{sec:ise}
%!TEX root=../paper.tex

\subsection{Variant 1 ($V_1$)}

\begin{itemize}

\INST{chacha20.v1.ad    rd, rs1, rs2}{
  uint32\xspace $a,b,c,d, ia, ib, ic,id, na,nd$; \;
  $a \ASN \GPR[*][{\VERB[ASM]{rs1}}] \RSH 32; d \ASN \GPR[*][{\VERB[ASM]{rs1}}];$
  $b \ASN \GPR[*][{\VERB[ASM]{rs2}}] \RSH 32; c \ASN \GPR[*][{\VERB[ASM]{rs2}}];$ \;
  $ia \ASN a + b; id \ASN (ia \XOR  d) \LRT 16;$;
  $ic \ASN c +id; ib \ASN (ic \XOR  b) \LRT12;$ \;
  $na \ASN ia+ib; nd \ASN (id \XOR na) \LRT 8;$ \;
  $\GPR[*][{\VERB[ASM]{rd}}] \ASN (na \LSH 32) \IOR nd;$ \;
}

\INST{chacha20.v1.bc    rd, rs1, rs2}{
  uint32\xspace $a,b,c,d, ib, ic,id, na,nd$; \;
  $a \ASN \GPR[*][{\VERB[ASM]{rs1}}] \RSH 32; d \ASN \GPR[*][{\VERB[ASM]{rs1}}];$
  $b \ASN \GPR[*][{\VERB[ASM]{rs2}}] \RSH 32; c \ASN \GPR[*][{\VERB[ASM]{rs2}}];$ \;
  $id \ASN (d \LRT 24) \XOR a; ic \ASN c \XOR id; ib \ASN (b \XOR ic) \LRT 12;$ \;
  $nc \ASN ic+ d; nb \ASN (ib \XOR nc) \LRT 7;$ \;
  $\GPR[*][{\VERB[ASM]{rd}}] \ASN (nb \LSH 32) \IOR nc;$ \;
}
\end{itemize}


\begin{algorithm}
\KwData  {64-bit values $X, Y$.
}
\KwResult{64-bit values $X, Y$.
}
\BlankLine
\KwFn{$\ALG{chacha\_qr}( X, Y )$}{
  \VERB[ASM]{chacha20.v1.ad} X, X, Y \;
  \VERB[ASM]{chacha20.v1.bc} Y, X, Y \;
  $\KwRet{\TUPLE{ X, Y }}$ \;
}
\caption{Chacha20's Quarter Round in Variant 1.}
\label{alg::qr::v1}
\end{algorithm}


\subsection{Variant 2 ($V_2$)}

\begin{itemize}

\INST{chacha20.v2.bd    rd, rs1, rs2}{
  uint32\xspace $a,b,c,d, nb,nd$; \;
  $a \ASN \GPR[*][{\VERB[ASM]{rs1}}] \RSH 32; d \ASN \GPR[*][{\VERB[ASM]{rs1}}];$
  $b' \ASN \GPR[*][{\VERB[ASM]{rs2}}] \RSH 32; d' \ASN \GPR[*][{\VERB[ASM]{rs2}}];$ \;
  $nd \ASN ((a + b) \XOR d) \LRT 16;$ 
  $nb \ASN ((c +id) \XOR b) \LRT 12;$ \;
  $\GPR[*][{\VERB[ASM]{rd}}] \ASN (nb \LSH 32) \IOR nd;$ \;
}

\INST{chacha20.v2.ad    rd, rs1, rs2}{
  uint32\xspace $a,b',d',d, nb,nd$; \;
  $a \ASN \GPR[*][{\VERB[ASM]{rs1}}] \RSH 32; d \ASN \GPR[*][{\VERB[ASM]{rs1}}];$
  $b \ASN \GPR[*][{\VERB[ASM]{rs2}}] \RSH 32; c \ASN \GPR[*][{\VERB[ASM]{rs2}}];$ \;
  $na \ASN b' + (d \XOR (d' \LRT 16));$; 
  $nd \ASN (na \XOR d') \LRT 8;$ \;
  $\GPR[*][{\VERB[ASM]{rd}}] \ASN (na \LSH 32) \IOR nd;$ \;
}

\INST{chacha20.v2.bc    rd, rs1, rs2}{
  uint32\xspace $a,b,c,d, t, na,nd$; \;
  $a \ASN \GPR[*][{\VERB[ASM]{rs1}}] \RSH 32; d \ASN \GPR[*][{\VERB[ASM]{rs1}}];$
  $b \ASN \GPR[*][{\VERB[ASM]{rs2}}] \RSH 32; c \ASN \GPR[*][{\VERB[ASM]{rs2}}];$ \;
  $t  \ASN c + (d \LRT 24) \XOR a;$ \;
  $nc \ASN t + d; nb \ASN (nc \XOR ((b \XOR t) \LRT 12)) \LRT 7;$ \;
  $\GPR[*][{\VERB[ASM]{rd}}] \ASN (nb \LSH 32) \IOR nc;$ \;
}
\end{itemize}

\begin{algorithm}
\KwData  {64-bit values $X, Y$.
}
\KwResult{64-bit values $X, Y$.
}
\BlankLine
\KwFn{$\ALG{chacha\_qr}( X, Y )$}{
  \VERB[ASM]{chacha20.v2.bd} Y, X, Y \;
  \VERB[ASM]{chacha20.v2.ad} X, X, Y \;
  \VERB[ASM]{chacha20.v2.bc} Y, X, Y \;
  $\KwRet{\TUPLE{ X, Y }}$ \;
}
\caption{Chacha20's Quarter Round in Variant 2.}
\label{alg::qr::v2}
\end{algorithm}


\subsection{Variant 3 ($V_3$)}


\begin{itemize}

\INST{chacha20.v3.ad0    rd, rs1, rs2}{
  uint32\xspace $a,b,d na,nd$; \;
  $a \ASN \GPR[*][{\VERB[ASM]{rs1}}] \RSH 32; b \ASN \GPR[*][{\VERB[ASM]{rs2}}] \RSH 32;$
  $d \ASN \GPR[*][{\VERB[ASM]{rs1}}];$ \;
  $na \ASN a + b; nd \ASN (na \XOR d) \LRT 16;$ \;
  $\GPR[*][{\VERB[ASM]{rd}}] \ASN (na \LSH 32) \IOR nd;$ \;
}

\INST{chacha20.v3.bc0    rd, rs1, rs2}{
  uint32\xspace $b,c,d, na,nd$; \;
  $b \ASN \GPR[*][{\VERB[ASM]{rs2}}] \RSH 32; c \ASN \GPR[*][{\VERB[ASM]{rs2}}];$
  $d \ASN \GPR[*][{\VERB[ASM]{rs1}}];$ \;
  $nc \ASN c + d; nb \ASN (nc \XOR b) \LRT 12;$ \;
  $\GPR[*][{\VERB[ASM]{rd}}] \ASN (nb \LSH 32) \IOR nc;$ \;
}

\INST{chacha20.v3.ad1    rd, rs1, rs2}{
  uint32\xspace $a,b,d na,nd$; \;
  $a \ASN \GPR[*][{\VERB[ASM]{rs1}}] \RSH 32; b \ASN \GPR[*][{\VERB[ASM]{rs2}}] \RSH 32;$
  $d \ASN \GPR[*][{\VERB[ASM]{rs1}}];$ \;
  $na \ASN a + b; nd \ASN (na \XOR d) \LRT  8;$ \;
  $\GPR[*][{\VERB[ASM]{rd}}] \ASN (na \LSH 32) \IOR nd;$ \;
}

\INST{chacha20.v3.bc1    rd, rs1, rs2}{
  uint32\xspace $b,c,d, na,nd$; \;
  $b \ASN \GPR[*][{\VERB[ASM]{rs2}}] \RSH 32; c \ASN \GPR[*][{\VERB[ASM]{rs2}}];$
  $d \ASN \GPR[*][{\VERB[ASM]{rs1}}];$ \;
  $nc \ASN c + d; nb \ASN (nc \XOR b) \LRT  7;$ \;
  $\GPR[*][{\VERB[ASM]{rd}}] \ASN (nb \LSH 32) \IOR nc;$ \;
}
\end{itemize}

\begin{algorithm}
\KwData  {64-bit values $X, Y$.
}
\KwResult{64-bit values $X, Y$.
}
\BlankLine
\KwFn{$\ALG{chacha\_qr}( X, Y )$}{
  \VERB[ASM]{chacha20.v3.ad0} X, X, Y \;
  \VERB[ASM]{chacha20.v3.bc0} Y, X, Y \;
  \VERB[ASM]{chacha20.v3.ad1} X, X, Y \;
  \VERB[ASM]{chacha20.v3.bc1} Y, X, Y \;
  $\KwRet{\TUPLE{ X, Y }}$ \;
}
\caption{Chacha20's Quarter Round in Variant 3.}
\label{alg::qr::v3}
\end{algorithm}


\subsection{Variant 4 ($V_4$)}

\begin{itemize}

\INST{chacha20.v4.add   rd, rs1, rs2}{
  uint32\xspace $a1,a0, b1, b0, r1,r0$; \;
  $a1 \ASN \GPR[*][{\VERB[ASM]{rs1}}] \RSH 32; a0 \ASN \GPR[*][{\VERB[ASM]{rs1}}];$
  $b1 \ASN \GPR[*][{\VERB[ASM]{rs2}}] \RSH 32; b0 \ASN \GPR[*][{\VERB[ASM]{rs2}}];$ \;
  $r1 \ASN a1 + b1; r0 \ASN a0 + b0;$ \;
  $\GPR[*][{\VERB[ASM]{rd}}] \ASN (r1 \LSH 32) \IOR r0;$ \;
}

\INST{chacha20.v4.xorrol.16  rd, rs1, rs2}{
  uint32\xspace $a1,a0, b1, b0, r1, r0$; \;
  $a1 \ASN \GPR[*][{\VERB[ASM]{rs1}}] \RSH 32; a0 \ASN \GPR[*][{\VERB[ASM]{rs1}}];$
  $b1 \ASN \GPR[*][{\VERB[ASM]{rs2}}] \RSH 32; b0 \ASN \GPR[*][{\VERB[ASM]{rs2}}];$ \;

  $r1 \ASN (a1 \XOR b1)\LRT 16; r0 \ASN (a0 \XOR b0)\LRT 16;$ \;
  $\GPR[*][{\VERB[ASM]{rd}}] \ASN (nb \LSH 32) \IOR nc;$ \;
}

\INST{chacha20.v4.xorrol.12  rd, rs1, rs2}{
  uint32\xspace $a1,a0, b1, b0, r1, r0$; \;
  $a1 \ASN \GPR[*][{\VERB[ASM]{rs1}}] \RSH 32; a0 \ASN \GPR[*][{\VERB[ASM]{rs1}}];$
  $b1 \ASN \GPR[*][{\VERB[ASM]{rs2}}] \RSH 32; b0 \ASN \GPR[*][{\VERB[ASM]{rs2}}];$ \;

  $r1 \ASN (a1 \XOR b1)\LRT 12; r0 \ASN (a0 \XOR b0)\LRT 12;$ \;
  $\GPR[*][{\VERB[ASM]{rd}}] \ASN (nb \LSH 32) \IOR nc;$ \;
}

\INST{chacha20.v4.xorrol.8  rd, rs1, rs2}{
  uint32\xspace $a1,a0, b1, b0, r1, r0$; \;
  $a1 \ASN \GPR[*][{\VERB[ASM]{rs1}}] \RSH 32; a0 \ASN \GPR[*][{\VERB[ASM]{rs1}}];$
  $b1 \ASN \GPR[*][{\VERB[ASM]{rs2}}] \RSH 32; b0 \ASN \GPR[*][{\VERB[ASM]{rs2}}];$ \;

  $r1 \ASN (a1 \XOR b1)\LRT 8; r0 \ASN (a0 \XOR b0)\LRT 8;$ \;
  $\GPR[*][{\VERB[ASM]{rd}}] \ASN (nb \LSH 32) \IOR nc;$ \;
}

\INST{chacha20.v4.xorrol.7  rd, rs1, rs2}{
  uint32\xspace $a1,a0, b1, b0, r1, r0$; \;
  $a1 \ASN \GPR[*][{\VERB[ASM]{rs1}}] \RSH 32; a0 \ASN \GPR[*][{\VERB[ASM]{rs1}}];$
  $b1 \ASN \GPR[*][{\VERB[ASM]{rs2}}] \RSH 32; b0 \ASN \GPR[*][{\VERB[ASM]{rs2}}];$ \;

  $r1 \ASN (a1 \XOR b1)\LRT 7; r0 \ASN (a0 \XOR b0)\LRT 7;$ \;
  $\GPR[*][{\VERB[ASM]{rd}}] \ASN (nb \LSH 32) \IOR nc;$ \;
}
\end{itemize}


\begin{algorithm}
\KwData  {64-bit values $X, Y, Z, W$.
}
\KwResult{64-bit values $X, Y, Z, W$.
}
\BlankLine
\KwFn{$\ALG{chacha\_hr}( X, Y, Z, W )$}{
  \VERB[ASM]{chacha20.v4.add}        X, X, Y \;
  \VERB[ASM]{chacha20.v4.xorrol.16}  W, W, X \;
  \VERB[ASM]{chacha20.v4.add}        Z, Z, W \;
  \VERB[ASM]{chacha20.v4.xorrol.12}  Y, Y, Z \;
  \VERB[ASM]{chacha20.v4.add}        X, X, Y \;
  \VERB[ASM]{chacha20.v4.xorrol.8}   W, W, X \;
  \VERB[ASM]{chacha20.v4.add}        Z, Z, W \;
  \VERB[ASM]{chacha20.v4.xorrol.7}   Y, Y, Z \;
  $\KwRet{\TUPLE{ X, Y }}$ \;
}
\caption{Chacha20's Half Round in Variant 4.}
\label{alg::hr::v4}
\end{algorithm}

% =============================================================================

\section{Evaluation}
\label{sec:res}
%!TEX root=../paper.tex

\subsection{Single block function performance}
\label{sec:eval:blk}
We evaluate the accelerated implementations of the ChaCha block function in software compared with a vanilla implementation used in OpenSSL~\cite{OpenSSL} as a Baseline. The four accelerated functions denoted as $V_1$, $V_2$, $V_3$, and $V_4$ use the corresponding ChaCha ISE variant sets to accelerate their round operations. 
Because ChaCha ISEs are encoded as R-type instructions, the accelerated functions easily use assembly RISC-V directives \VERB[asm]{.insn} to invoke ChaCha ISEs without the requirement of building modified toolchains.   
For comparison's sake, all accelerated and baseline functions have the same function prototype and looping scheme (i.e. 10 double rounds each of which includes an odd and an even rounds). 
They are complied with the RISC-V gcc version 9.2.0 with a performance optimisation flag (`-02') and targeting for the \VERB{rv64imac} architecture (`-march=rv64imac -mabi=lp64'). 
We use an open-source Verilator tool~\cite{Verilator} compiling the Verilog files of the integrated system (mentioned in \REFSEC{sec:ise:hw:sys}) including a Rocket core and ChaCha ISEs to generate a cycle-accurate behavioural emulator for the system. The generated emulator is used to evaluate the performance of the software of the ChaCha block functions.  

\begin{table}[b]
\caption{Comparison of chacha block function performance}
\label{tab:res:sw:perf1}
\begin{tabular}{crrl}
\toprule            
Implementations        & Inst. count   & Cycle count & Inst. footprint\\

\midrule
Baseline     & 2214 ($1.00\times$)  & 2803 ($1.00\times$)    &  852 ($1.00\times$)  \\
 $V_1$ &  434 ($0.20\times$)     & 1221 ($0.44\times$) &  382 ($0.45\times$) \\
 $V_2$ &  514 ($0.23\times$)     & 1597 ($0.57\times$) &  428 ($0.50\times$)\\
 $V_3$ &  594 ($0.27\times$)     & 1972 ($0.70\times$) &  454 ($0.53\times$)\\
 $V_4$ &  464 ($0.21\times$)     & 1818 ($0.65\times$) &  274 ($0.32\times$)\\

\bottomrule
\end{tabular}
\end{table}

The evaluation is shown in \REFTAB{tab:res:sw:perf1} in terms of instruction count, cycle count, and instruction footprint (in bytes) of the ChaCha block function. As can be seen, the accelerated functions have significantly reduced instruction counts to about 20\% of the baseline instruction count. Even though the cycle counts are also reduced in the case of the accelerated functions, but the reduction in cycle count metrics is not as good as in instruction count metrics. This could be due to inefficient data forwarding operations supporting ChaCha ISEs via the RoCC interface in the Rocket core micro-architecture. It should be noted that the operation of every ChaCha ISEs is computed in one clock cycle. One can view that as a trade-off between ineffective performance and invasiveness avoiding micro-architectural modifications.

Comparing the ISE variants, $V_4$ obtains a good trade-off solution which provides the lowest instruction footprint and the second-lowest instruction count (32\% and 21\%, respectively, compared to the baseline) and consumes a small area overhead (see \REFTAB{tab:res:hardcost1}).

\subsection{Encryption/Decryption performance}
We then implement a completed ChaCha encryption/decryption function in which the accelerated block functions implemented in the above subsection is used to generate keystream blocks. The keystream blocks xors with input data-stream blocks to encrypt/decrypt the data streams. We invest the $V_4$ set of the proposed ISE to accelerate the encryption function. 

The performance of the implementation using the proposed ISE is evaluated in comparison to the existing implementations including scalar and vectorisation implementations.  
For scalar (no vectorisation) implementations, only the standard 64-bit ISA (scalar) instructions of RISC-V are used. We choose the ChaCha implementation of OpenSSL as the Baseline. In addition, we implement an optimised variant denoted \VERB{Opt.1} which is written in assembly language to optimise the performance with our best effort. Moreover, we invest an optimised implementation denoted \VERB{Opt.2} to use the Bitmanip extension supporting rotation instructions.

For the vectorisation implementations, we make use of vectorised instructions to accelerate ChaCha encryption/decryption operations. 
We adopt two approaches, one, denoted \VERB{Vector1}, implements a cell-oriented approach used in OpenSSL which processes multiple blocks in parallel.
The other, denoted \VERB{Vector2}, follows a row-oriented approach presented in~\cite{GolGue:14}. This approach packs state elements in ChaCha blocks' rows into the same vectorised registers.
In addition, we invest two versions of vector lengths, namely 128 bits and 256 bits, for the vectorisation implementations. 
Different from the original implementations, the vectorisation implementations are realised using the vector instruction extension set~\cite{riscv:ext:vector:draft} for RISC-V processors instead of using AVX/AVX2 architecture on \VERB{x86_64} processors.

The implementations are compiled as the same set up in \REFSEC{sec:eval:blk} but targeting to the \VERB{rv64imacb} and the \VERB{rv64gcv} architectures for the Bitmanip extension and the vector instruction extension, respectively. 
Currently, the Bitmanip and the vector instruction extensions have not been frozen. We adopt the latest published versions v0.92 and v0.9 for the Bitmanip and the vector extensions, respectively. Since there is not yet an open-source implementation supporting the RISC-V vector instruction extension is available, we use \VERB{Spike}~\cite{Spike}, an instruction set simulator, to evaluate the instruction count of the implementations.

\begin{table*}
\caption{Comparison of encryption/decryption performance in instruction count for different message sizes between the Baseline, ISA-based optimised implementation, ISE-assisted implementation and different vectorization implementations}
\label{tab:res:sw:perf2}
\begin{tabular}{rrrrrrrrr}
\toprule             
Message size & Baseline  &  RV64I & RV64IB  &  ISE   & \multicolumn{2}{c}{128 bit Vector} & \multicolumn{2}{c}{256 bit Vector} \\
             & OpenSSL   &  Opt.1 &  Opt.2  & $V_4$  & \VERB{Vector1} & \VERB{Vector2}    & \VERB{Vector1} & \VERB{Vector2}    \\
\midrule
  64 bytes   &    2825   &  1768 &  1129 &  523   &    2001        &       607         &    2001        &       615         \\
 128 bytes   &    5555   &  3486 &  2207 &  989   &    2001        &      1182         &    2001        &       615         \\
 256 bytes   &   11015   &  6922 &  4363 & 1921   &    2001        &      2332         &    2001        &      1191         \\
 512 bytes   &   21935   & 13794 &  8675 & 3785   &    3748        &      4632         &    2001        &      2343         \\
1024 bytes   &   43775   & 27538 & 17299 & 7716   &    7242        &      9232         &    3748        &      4647         \\
\bottomrule 
\end{tabular}
\end{table*}

\REFTAB{tab:res:sw:perf2} reports the instruction count executing encryption/decryption operations for different message sizes. 
As expected, all accelerated implementations including the cases of scalar ISE, vectorised ISE gain significant reductions in instruction count compared to the baseline. 
We observe that the lack of supporting rotation in the current version of the vector instruction extension introduces the disadvantages of the implementations based on this extension. 
However, for the large messages, the vector-based implementations show their advantage over the scalar ISE based implementation. 
The \VERB{Vector1} implementations provide the lowest instruction count executions when the message size is greater than 512 bytes. 
The \VERB{Vector2} implementations outperform the \VERB{Vector1} implementations for shorter messages.
Interestingly, the proposed scalar ISE-assisted implementation, $V_4$, gain the best performance in the case of single block messages.
For the message size smaller than 512 bytes, the $V_4$ implementation have a better performance compared to the \VERB{Vector1} implementation. 
And $V_4$ outperforms the 128-bit Vector implementation of \VERB{Vector2} for all message sizes. But when the vector length increases to 256-bit, \VERB{Vector2} shows its advantage over $V_4$. It is worth noting that the vector instruction extension which is only presented in high(er)-end computational platforms cause a very large overhead in hardware while the proposed ISE approach requires negligible increased hardware cost to gain a good performance for short messages compared to the vectorisation implementations. That makes our ChaCha ISE be suitable for low(er)-end resource-constrained processors.

In comparing the scalar implementation, the \VERB{Opt.1} implementation gains a reduced instruction count to 63\% of the OpenSSL baseline instruction count.
It obtains an encrypting performance of $26.9$ instructions/byte (with 1024 byte message) that is almost similar to the result reported in~\cite{Sto:19} (i.e. $27.9$ cycles/byte with most instructions executed in a single cycle). Moreover, the \VERB{Opt.2} implementation has further improvement that reduces its instruction count to 40\% of the baseline instruction count.
Notably, the instruction count of the $V_4$ implementation is reduced to at least 19\% (resp. 29\%) of the baseline (resp. the \VERB{Opt.1}) instruction count. In other words, the $V_4$ implementation achieves a $5.4\times$ (resp. $3.4\times$) speed-up compared to the baseline (resp. the \VERB{Opt.1}) implementation. 



% =============================================================================

\section{Conclusion}
\label{sec:outro}
%!TEX root=../paper.tex
In this paper, we presented the design, implementation, and evaluation of three ISE variants to support ChaCha stream cipher on 64-bit RISC-V architecture.
We show that our proposed ISE variants help improve significantly the efficiency of ChaCha block function wrt. both execution latency and memory footprint.  
Our ISE-assisted implementation of ChaCha encryption speeds up at least $5.4\times$ and $3.4\times$ compared to the OpenSSL baseline and ISA-base optimised implementation, respectively. 
For encrypting short messages, the ISE-assisted implementation gains a comparative performance compared to the vectorised implementations which demand a large hardware overhead to support vector instruction extension.
Moreover, the ISE hardware implementation only causes a negligible increased area overhead, about 3\%, on the Rocket Chip system that makes our ChaCha ISE be suitable for resource-constrained processors. While showing potential, the proposed ChaCha ISE needs further investigations in future works to be more generic and efficient for other ARX ciphers.






% =============================================================================


\ifbool{ACMsubmission}{%
\bibliographystyle{ACM-Reference-Format}
}{%
\bibliographystyle{alpha}
}%
\bibliography{paper}

% =============================================================================

\appendix

\newpage
%\onecolumn
\section{The ISE-assisted Implementation of the block function}
\label{appx:ISE-assisted Implementation}
\lstset{style=mystyle}
\lstinputlisting[language=C]{./tex/blockfunction.c}
%\import{./tex/}{ise-algorithm-conv.tex}

\end{document}


